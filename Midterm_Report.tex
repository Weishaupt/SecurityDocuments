\documentclass[a4paper,draft]{scrartcl}

\usepackage[english]{babel} %English 
%\usepackage[T1]{fontenc}
\usepackage[utf8]{inputenc}	%utf8 encoding
\usepackage[colorlinks]%, %Anstatt Boxen um Links, diese einfärben
        	{hyperref}		%Anklickbare Links

%\usepackage{
%	amsmath,                 % improves quality of formulas
%   amssymb,                 % mathematical symbols
%   amsfonts,                % mathematical fonts
%   amsthm,                  % macros for theorems, lemmas, etc.
%   graphicx,                % needed to include graphics and images
%}

\newcommand{\totype}{\(\to\;\)}

\title{KeyChain Extension and Integration\\
	Midterm Report}
\subtitle{Practical Lab on Smartphone Security}
\author{Kjell Braden, Marvin Dickhaus, Cassius Puodzius}
\date{Winter term 2012/2013}


%%% START OF DOCUMENT %%%%%%%%%%%%%%%%%%%%%%%%%%%%%%%%%%%%%%%%%%%%%%%%%%%%%%%%

\begin{document}

\maketitle

\begin{abstract}
	The goal of this project is to extend the build-in key storage of Android 4.2 in a way that it supports functionality such as signing and decrypting without revealing the needed key to any app.\\
	The second goal is the integration of system apps like Contacts or E-Mail with the newly provided features.
\end{abstract}

\tableofcontents

\section{Introduction}
	Since Android API Level\footnote{\url{http://developer.android.com/guide/topics/manifest/uses-sdk-element.html\#ApiLevels}} 14 (4.0 Ice Cream Sandwich) the KeyChain\footnote{\url{http://developer.android.com/reference/android/security/KeyChain.html}} class exists in Android. The KeyChain class provides access to private keys and their corresponding certificate chains in credential storage.

	Whenever in the current implementation an authorized add calls to retrieve the private key, it'll get it. Our goal is to extend the given functionality further with a crypto oracle. With that functionality, apps should call the API with the data they want en-/decrypted or signed/verified. In the process of en-/decryption and signing the API would return the respective string. With verifying the API would return a boolean indicating verification success.

	Besides the new API, we want to alter at least one system app to comply with our enhanced KeyChain.

\section{Initial Situation}
	At the starting point, we were confronted with a quite unfinished, not wholly tested API. Furthermore the KeyChain API lacks in documentation. So we had to figure out, in what way it would be possible to implement our functionality.

	\subsection{Class Overview}
		\begin{description}
			\item[KeyChain] is the Android API for importing PKCS\#12 containers (private keys, public key certificates and CA certificates) and providing grant-based access to the keys to apps. The PKCS\#12 format allows the container to be encrypted and/or signed.
				\begin{itemize}
					\item when an import is requested the API creates an Intent, which is handled in the system app \path{com.android.certinstaller} (source at \path{packages/apps/CertInstaller/CertInstaller.java} et al.). CertInstaller spawns a dialog and handles the decryption process as well as stores the keys in the \texttt{keystore} daemon (see below).
					\item Keys are identified by aliases, which are chosen by the user when the CertInstaller is invoked with an import request.
					\item Private as well as public keys (certificates) can be requested:
					\begin{itemize}
						\item If access to a private key was granted to the requesting app, it can be retrieved using \texttt{getPrivateKey()}.
						\item Public keys can be retrieved through \texttt{getCertificateChain()}
					\end{itemize}
				\item Also contains \texttt{AndroidKeyPairGenerator} since Android 4.2, which generates and stores key pairs in \texttt{key\-store} automatically.
				\end{itemize}

			\pagebreak[3]
			\item[keystore] The \texttt{keystore} is the native deamon (written in C) that holds encrypted key information.
				\begin{itemize}
					\item The storage is encrypted with a master key which is derived from the unlock passphrase, PIN or pattern.
					\item Unlocks on the \emph{first successful unlock attempt} of the device, won't lock again \emph{until the phone is powered off}. This means the keystore is protected for example from rooting, but not from live-debugging. (It is possible for apps to lock it manually, though.)
					\item The service supports two storage types:
					\begin{itemize}
						\item[key] is a RSA keypair. Once stored the private key cannot be exported again, more on this later.
						\item[blob] can be arbitrary data.
					\end{itemize}
					\item \texttt{keystore} provides an OpenSSL engine called \texttt{keystore}, which should be able to load private keys. %As I see it, this actually receives PKCS#8 containers, doesn't it?
				\end{itemize}
			\item [CertInstaller] The \texttt{CertInstaller} is a system app that lets a user import and install key pairs and certificates from PKCS\#12 container format files. %TODO: Expand!
		\end{description}






	\subsection{Proposed Implementation Ideas}
		Before we could get any real work done, we needed an angle on how we could implement our enhancements.

		Using \texttt{keystore}'s own \texttt{key} storage format sounded promising, as it already applies non-exportability and provides sign/verify operations.
		\subsubsection{AndroidKeyPairGenerator \& KeyChain}
			At first we tried to reuse as much code as possible.		
			\begin{enumerate}
				\item Use \path{android.security.AndroidKeyPairGenerator} for key generation and storage.
				\item Use \path{KeyChain.getCertificateChain()} and \path{java.security.Cipher} for encryption.
				\item Use \path{KeyChain.getPrivateKey()} and \path{java.security.Cipher} for decryption.
			\end{enumerate}
			The results were rather frustrating. \texttt{AndroidKeyPairGenerator} isn't really well developed as of yet.
			\begin{itemize}
				\item It's not registered as a \texttt{java.security.KeyPairGenerator} provider, so we either have to change the ROM or each app will have to register it on itself.
				\item It's not configurable in terms of private key parameters (key size, key type\footnote{At this point, the key type is hard coded to RSA by the \texttt{keystore} anyway. While we could store the keys as blobs to support more key formats, the current Android \texttt{java.security.Security} providers only support RSA for asymmetric encryption and RSA and DSA for signatures, so we wouldn't gain much.})
			\end{itemize}
			In conclusion: While we could live with the rest, fixed key size is not quite what we were hoping for. Also, for some reason, \path{KeyChain.getPrivateKey()} does not work - the \texttt{keystore} fails with \texttt{KEY\_NOT\_FOUND}.



		\subsubsection{KeyPairGenerator \& CertInstaller}
			When importing a PKCS\#12 file the \texttt{CertInstaller} does the following:
			\begin{enumerate}
			\item ask the user for the container's password
			\item decrypt the container, parse contents to \texttt{java.security.PrivateKey}
			\item ask the user to provide an alias for further identification of the key pair
			\item re-encode the keys
			\item send to \texttt{com.android.settings.CredentialStorage}, which sends the key to the \texttt{keystore} daemon for storage
			\end{enumerate}
			
			Instead of using the Android specific \texttt{KeyPairGenerator}, we could use the default \texttt{KeyPairGenerator}\footnote{\url{http://docs.oracle.com/javase/6/docs/api/java/security/KeyPairGenerator.html}} provided by Java and send the key pair to \texttt{CertInstaller} to handle the storage.
			
			Unfortunately, we were unable to skip the first two steps - we would have had to provide our keys as encrypted PKCS\#12 containers. Since we already had the contents of the container (because we were generating it with \texttt{KeyPairGenerator}), it wouldn't make sense to pack and encrypt them, just to decrypt and unpack them again.
			
			Thus we implemented step 3 to 5 on our own and let \texttt{CredentialStorage} take care of the storage\footnote{For this we had to move our code from the \texttt{KeyChain} app to the \texttt{CertInstaller} app because the CredentialStorage refuses to talk to anybody else.}, this left us with working key generation and storage.
			
			% TODO i don't like this phrasing
			The problem now is, that \path{KeyChain.getPrivateKey()} won't work. It turns out: \texttt{getPrivateKey()} uses prepares the alias by checking some permissions and adding the system uid (1000) to the name. In the end the \texttt{keystore} daemon tried to read \texttt{1000\_1000\_USRPKEY\_alias} instead of \texttt{1000\_USRPKEY\_alias}.

			Conclusion: We can't use \texttt{KeyChain.getPrivateKey()} either.

		\pagebreak[4]
		\subsubsection{CertInstaller \& OpenSSL engine}
			The keystore offers a custom OpenSSL engine which provides access to \texttt{PrivateKey} objects backed by the storage.

			\begin{enumerate}
				\item Use our code in \texttt{CertInstaller} as outlined above.
				\item For decryption, load private key directly from the \texttt{keystore} OpenSSL engine (like \texttt{KeyChain} does).
				\item For encryption, load public key using \path{KeyChain.getCertificateChain()}
			\end{enumerate}

			Observation:
			The \emph{decryption crashes the process} (SIGSEGV, not a java exception)

			Conclusion:
			texttt{keystore}'s \texttt{key} type is broken. The \texttt{keystore} only allows access to the public key, it seems to returning the public key even when a private key was requested. The included OpenSSL Java-API parses this as a \texttt{PrivateKey} and returns this. Any privatekey-crypto operation obviously crashes.
		
		\subsubsection{Do everything manually}
			Obviously none of the above ideas where very fulfilling. As a result we're going to do everything in our own code and store the keys as blobs in the \texttt{keystore}. That means:
			\begin{enumerate}
				\item Manually generate key, manually store it in \texttt{keystore} as PKCS\#8 (not PKCS\#12) encoded blob and certificate as PEM encoded blob.
				\item For encryption, manually load certificate from \texttt{keystore}, parse it, feed to \path{java.security.Cipher}.
				\item For decryption, manually load private key from \texttt{keystore}, parse it, feed to \path{java.security.Cipher}.
			\end{enumerate}
			As a result, this is the only working constellation we could acquire.

\section{Implementing the framework}
	Once we decided to do this manually, we needed to determine how our interface should look. To reduce responsibility of the apps it makes sense to let the system app (\texttt{KeyChain}) handle everything related to key management:
	\begin{enumerate}
		\item Generate keys
		\item Delete keys
		\item Import/Export of key {\em pairs}\footnote{Users could export their keys for use on other devices.}
		\item Grant key access to apps
	\end{enumerate}

	We need the following functionality to be available to our app:
	\begin{enumerate}
	\tt
		\item encrypt(keyId, plaintext, blockCipherMode) \totype ciphertext
		\item decrypt(keyId, ciphertext, blockCipherMode) \totype plaintext
		\item sign(keyId, plaintext) \totype signature
		\item verify(keyId, plaintext, signature) \totype isValid
		\item requestKey() \totype keyId
		\item storePublicKey(keyId, publicKey) \totype void
		\item getPublicKey(keyId) \totype PublicKey
	\end{enumerate}
	Keys itself are referenced by an string alias.

	On \texttt{requestKey()}, the \texttt{KeyChain} app displays a key selection dialog to the user, informing her which app requested it and giving her the option to generate a new key. Once confirmed, the requesting app is granted usage of the selected key.

	We have implemented this API and an userspace app to test this basic functionality.

\section{Perspective - or - Extend the system apps}
In our further planning and development, we want to integrate the contacts, and stock e-mail app with native support for our crypt oracle. The steps therefor have yet to be evaluated.

\end{document}

