%\documentclass[a4paper]{scrartcl}
\documentclass[a4paper,draft]{scrartcl}

\usepackage[english]{babel} %English 
%\usepackage[T1]{fontenc}
\usepackage[utf8]{inputenc}	%utf8 encoding
\usepackage[colorlinks]%, %Anstatt Boxen um Links, diese einfärben
        	{hyperref}		%Anklickbare Links

\interfootnotelinepenalty=10000 % this prevents footnotes from being split across multiple pages
%\usepackage{
%	amsmath,                 % improves quality of formulas
%   amssymb,                 % mathematical symbols
%   amsfonts,                % mathematical fonts
%   amsthm,                  % macros for theorems, lemmas, etc.
%   graphicx,                % needed to include graphics and images
%}

\newcommand{\totype}{\(\to\;\)}

\title{KeyChain Extension and Integration\\
	Final Report}
\subtitle{Practical Lab on Smartphone Security}
\author{Kjell Braden, Marvin Dickhaus, Cassius Puodzius}
\date{Winter term 2012/2013\\\today}


%%% START OF DOCUMENT %%%%%%%%%%%%%%%%%%%%%%%%%%%%%%%%%%%%%%%%%%%%%%%%%%%%%%%%

\begin{document}

\maketitle

\begin{abstract}
	The goal of this project is to extend the built-in key storage of Android 4.2 to support functionality such as signing and decrypting without revealing the needed key to any app.\\
	The second goal is the integration of system apps to demonstrate the functionality. This group implemented symmetrically encrypted SMS for that purpose.
\end{abstract}

\tableofcontents
\pagebreak[4]

\section{Introduction}
	Since Android API Level\footnote{\url{http://developer.android.com/guide/topics/manifest/uses-sdk-element.html\#ApiLevels}} 14 (4.0 Ice Cream Sandwich) the KeyChain\footnote{\url{http://developer.android.com/reference/android/security/KeyChain.html}} class exists in Android. The KeyChain class allows storage of asymmetric keys in a secure credential storage.

	Whenever in the current implementation an authorized app calls to retrieve the private key, it'll get it. Our goal was to extend the given functionality further with a crypto oracle, as well as support for different key types, both symmetric and asymmetric. With that functionality, apps should call the API with the data they want en-/decrypted or signed/verified. In the process of en-/decryption and signing the API would return the respective byte data. With verifying the API would return a boolean indicating verification success or failure.
	
	Besides the new API, we wanted to alter at least one system app to comply with our enhanced KeyChain. After suggestions from Sven Bugiel (our supervisor) we decided to implement symmetric encryption with the SMS app in lieu of the E-Mail app. More on that in section \ref{sec-sms-app}.

\section{Initial Situation}
	At the starting point, we were confronted with a quite unfinished, not wholly tested API. Furthermore the KeyChain API lacks in documentation. So we had to figure out in what way it would be possible to implement our functionality.

	\subsection{Class Overview}
		\begin{description}
			\item[KeyChain] is the Android API for importing PKCS\#12 containers (private keys, public key certificates and CA certificates) and providing grant-based access of the keys to apps. The PKCS\#12 format requires the container to be encrypted.
				\begin{itemize}
					\item When an import is requested the API creates an Intent, which is handled in the system app \path{com.android.certinstaller} (source at \path{packages/apps/CertInstaller/CertInstaller.java} et al.). CertInstaller spawns a dialog and handles the container decryption process as well as stores the keys in the \texttt{keystore} daemon (see below).
					\item Keys are identified by aliases, which are chosen by the user when the CertInstaller is invoked with an import request.
					\item Private as well as public keys (in the form of certificates) can be requested:
					\begin{itemize}
						\item If access to a private key was granted to the requesting app, it can be retrieved using \texttt{getPrivateKey()}.
						\item Public keys can be retrieved through \texttt{getCertificateChain()}
					\end{itemize}
				\item \texttt{KeyChain} also contains \texttt{AndroidKeyPairGenerator} since Android 4.2, which generates and stores key pairs in \texttt{key\-store} automatically.
				\end{itemize}

			\item[KeyChain app] is a system application implementing most parts of the \texttt{KeyChain} API. It grants key access to apps and retrieves keys from the secure native \texttt{keystore} daemon (see below).

			\item[keystore] is the native deamon (written in C++) that holds encrypted key information.
				\begin{itemize}
					\item The storage is encrypted with a master key which is derived from the unlock passphrase, PIN or pattern of the device\footnote{In order to import and use certificates that are not in the trusted certificates from the Android base (known as user certificates), respectivey use the \texttt{keystore}, a PIN, passphrase or pattern has to be created, that'll unlock the device.}.
					\item Unlocks on the \emph{first successful unlock attempt} of the device, won't lock again \emph{until the phone is powered off}. This means the keystore is protected for example from rooting, but not from live-debugging. (It is possible for apps to lock it manually, though.)
					\item The service supports two storage types:
					\begin{description}
						\item[key] is a RSA keypair. Once stored the private key cannot be exported again, more on this later.
						\item[blob] can be arbitrary data.
					\end{description}
					\item \texttt{keystore} provides an OpenSSL engine called \texttt{keystore}, which should be able to retrieve PrivateKey objects from the daemon directly in Java code.
				\end{itemize}
			\item [CertInstaller] The \texttt{CertInstaller} is a system app that lets a user import and install key pairs and certificates from PKCS\#12 container format files. %TODO: Expand!
		\end{description}

	\subsection*{Detour: Java Cryptography Architecture}
		The JCA is a {\em provider framework} for common cryptography operations such as hashing, sign/verify,
		encrypt/decrypt and key/certificate generation. Each of these operations have an algorithm- and
		implementation-independent interface, most of which operates on objects implementing the
		\path{javax.crypto.SecretKey}, \path{java.security.PublicKey} or \path{java.security.PrivateKey}
		interfaces, depending on the context.

		This makes it possible for crypto code to be written largely independent on actual implementation
		e.g. of the keys.

	\section{Requirements}
		To achieve our goal, we need to be able to securely store and generate both symmetric and asymmetric
		keys, regardless of the key type.

		Apps should be able to do crypto operations with specific keys, access to which should require the user
		to give her permission. For private keys, these operations include decryption and signing of data, while
		for public keys it's encryption and verification. With symmetric keys, apps need to be able to both
		encrypt and decrypt, as well as sign and verify (which means generating and checking {\em message
		authentication codes} in this case).

		These operations must not leak any kind of private key parameter or any part of the symmetric key used.

		For convenience, it would be a good idea to create and store key exchange (eg. Diffie-Hellman) parameters
		as well, and to automatically derive and store the symmetric key on receiving the remote's parameters.

	\subsection{Workflow with the existing KeyChain API}
		In the KeyChain API there is \texttt{AndroidKeyPairGenerator}, \texttt{AndroidKeyStore} and \linebreak \texttt{KeyChain}.

		The \texttt{AndroidKeyPairGenerator} implements JCA's \texttt{KeyPairGenerator} interface and sends a request
		to generate a key to the {\em keystore} daemon. This way, the keys' types are hard coded to {\em 2048 bit RSA}.
		
		The \texttt{AndroidKeyStore} implements the \texttt{KeyStore} interface. It retrieves the key references from
		the daemon, but does no permission checking. If the requesting app has never been granted to access the requested
		key, the access simply fails. It supports only asymmetric keys as well.

		Key references means that the returned key objects don't carry the parameters themselves, but instead they are
		handles that the OpenSSL JCA provider can use to run the crypto operations on the keys in the keystore, so the
		Java code (and specifically, the app's code) never sees any crypto parameters at all.

		The \texttt{KeyChain} allows to retrieve \texttt{PrivateKey}s using the \texttt{AndroidKeyStore}, but requests
		permission to access a key beforehand. The user will see a dialog telling him which app tries to access a key,
		she can choose the one she wants to use and the app gets access to that specific key.

	\subsection{Proposed Implementation Ideas}
		\subsubsection{Reuse KeyChain API}
			Our initial plan was to reuse as much of the KeyChain API as possible. Nevertheless we would need to add support
			for symmetric keys and other asymmetric key types in both storage and key generation as well as support for key
			exchange handling.

			This turned out to be harder than we thought, as we would need to rewrite significant amount of C++ code in the
			\texttt{keystore} daemon for supporting anything else than RSA. Also we would have to rewrite the protocol the
			\texttt{keystore} uses to communicate with the Java framework.

			And finally, there turns out to be a bug somewhere in Android's OpenSSL engine, which causes {\em dalvik} to
			{\em segfault} as soon as encryption or decryption is being run with the returned key references.

		\subsubsection{The manual way}
			% TODO rewrite to match the above
			Obviously none of these approaches worked out very well. As a result we're going to do everything in our own code and store the keys as blobs in the \texttt{keystore}. This means:
			\begin{enumerate}
				\item Manually generate the key pair, manually store it in the \texttt{keystore} as a PKCS\#8 (not PKCS\#12) encoded blob and the certificate as PEM encoded blob.
				\item For encryption, manually load the certificate from \texttt{keystore}, parse it, feed it to \path{java.security.Cipher}.
				\item For decryption, manually load the private key from \texttt{keystore}, parse it, feed it to \path{java.security.Cipher}.
			\end{enumerate}
			As a result, this is the only working constellation we could acquire.

\section{Implementing the framework}
	Once we decided how to generate and store the keys, we needed to determine how our interface should look. To reduce responsibility of the apps it makes sense to let the system app (\texttt{KeyChain}) handle everything related to key management:
	\begin{enumerate}
		\item Generate keys
		\item Delete keys
		\item Import/Export of key \emph{pairs}\footnote{Users should be able to export their keys for use on other devices.}
		\item Grant key access to apps
	\end{enumerate}
	Our first sketch didn't quite reflect symmetric encryption, so there was some more work due in refining our API, with the final result being as followed.

	% TODO match completely to current API
	\begin{enumerate}
	\tt
		\item encrypt(keyId, algorithm, padding, plaintext, initVector) \totype ciphertext
		\item decrypt(keyId, algorithm, padding, ciphertext, initVector) \totype plaintext
		\item sign(keyId, algorithm, plaintext) \totype signature
		\item verify(keyId, algorithm, plaintext, signature) \totype isValid
		\item requestKey() \totype keyId
		\item storePublicCertificate(keyId, certificate) \totype void\footnote{Existing methods in \texttt{android.security.KeyChain} can be used for retrieving the certificate / public key}
		\item generateSymmetricKey(keyId, algorithm, keysize) \totype void
		\item retrieveSymmetricKey(keyId, algorithm) \totype key
		\item importSymmetricKey(keyId, key) \totype void
		\item deleteSymmetricKey(keyId) \totype void
		\item mac(keyId, algorithm, plaintext) \totype mac
		\item generateKeyPair(keyId, algorithm, keysize) \totype publicKey
	\end{enumerate}
	Keys itself are referenced by an string alias.

	% TODO explain assignment of contacts/key usage types/key aliases

\section{Integrating the SMS app with our API}
	\label{sec-sms-app}
	At first we thought about integrating asymmetric cryptography with the standard e-mail app. As development moved on, we decided to go with implementing symmetric cryptography with the SMS app instead. % TODO why?
	In our scenario we assumed, that the symmetric key would already be exchanged with the other party. So we just had to handle sending and receiving SMS.
	
	The greatest inconvenience with system apps is, that you first have to build the image, once you've developed something. As compiling android even on good systems took
	
\section{Responsibilities}
	%Who did what?
	%Reports - All of us
	%Backend - Kjell, (Cassius?)
	%Crypt-SMS-App - All of us with Kjell doing the most?
	%Key-management app - Kjell
	In general the project can be divided in the backend and the frontend. The backend being the CryptOracle and 
	
\section{Further development}
	%What could come next?

\end{document}

